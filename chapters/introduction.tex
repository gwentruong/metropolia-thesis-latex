% Introduction

\chapter{Introduction}

Minimizing fluid energy losses is one of important missions in hydraulic engineering and chemical engineering to improve the overall efficiency of the system. This loss of energy is usually caused by viscosity, change of velocity in pipe.

The new product from SansOx Ltd is an economic solution to this life-long problem. The product is called Voxer, which consists of a Voxer wing and a tube. The wing creates vortex flow that reduces turbulence and flow loss in the tube. Voxer can be installed in various places allocated accordingly in a whole pipeline. 

The objectives of this thesis is to present Voxer's function in reducing energy consumption in pipe and to examine the affect of different Voxer wing's combinations. The thesis was also carried out to analyze the hydraulic phenomenon and determine the Voxer's performance when introducing to the market. \newline
The experiments were conducted in Keravan Bio-power plant of Keravan Energy Limited. Keravan Energy Ltd. is a state owned company in Kerava and Sipoo municipality. They produce electricity and district heating to not only serve their own municipality but also to sell electricity to the whole Finland. The object of this study is the water cooling piping system of the power plant.

SansOx Limited is the commissioner of this project. SansOx Ltd. focuses on researching, developing and marketing fresh innovation solutions for clean water market worldwide. Their goal is to advanced products in order to provide the optimal sustainable solutions for customers' needs in water treatment, process wastewater treatment, fish farming and agriculture. By executing this study, we will contribute to a completed theory profile of Voxer with quantitative research regarding to hydraulic and geometrical parameters of industrial piping system. 

In this thesis, we will demonstrate the experimental fluid dynamics setup and analysis on Voxer under laboratory condition and realistic condition in the power plant. Also, the limitation and possible further study will be discussed. 

\clearpage %force the next chapter to start on a new page. Keep that as the last line of your chapter!
