% Introduction

\chapter{Introduction}

Minimizing fluid energy losses is one of the important missions in hydraulic engineering and chemical engineering to improve the overall efficiency of the system. The flow resistance in a pipe is caused by various reasons, such as \gls{viscos}, pipe roughness or change of velocity.

The new product from \gls{ltd} is an economic solution to this problem. The product is called Voxer, which consists of a Voxer wing to be inserted into a tube. The wing creates vortex flow that reduces turbulence and flow losses in the tube. Voxer can be installed in various places allocated accordingly in a whole pipeline.

The objectives of this thesis are to present Voxer’s function in reducing energy consumption in the pipe and to examine the effect of different Voxer wing’s combinations. The thesis was also carried out to analyze the hydraulic phenomenon and determine the Voxer’s performance when introducing to the market.

The experiments were conducted in Keravan Bio-power plant of Keravan Energy Limited. Keravan Energy Ltd. is a state-owned company in Kerava and Sipoo municipality. They produce electricity and district heating to not only serve their own municipality but also to sell electricity to the whole of Finland. The object of this study is the water cooling piping system of the power plant.

SansOx Limited is the commissioner of this project. \gls{ltd} focuses on researching, developing and marketing fresh innovation solutions for clean water market worldwide. Their goal is to develop products in order to provide the optimal sustainable solutions for customers’ needs in water treatment, process wastewater treatment, fish farming and agriculture. By executing this study, we will contribute to a completed theory profile of Voxer with quantitative research regarding hydraulic and geometrical parameters of an industrial piping system.

In this thesis, we will demonstrate the experimental fluid dynamics setup and analysis on Voxer under laboratory condition and realistic condition in the power plant. Also, the limitation and possible further study will be discussed.
\clearpage %force the next chapter to start on a new page. Keep that as the last line of your chapter!
