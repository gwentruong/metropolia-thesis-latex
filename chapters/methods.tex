% Material and Methods

\chapter{Methodology}

There are two aspects of this thesis project. The first aspect is to perform a comparative testing to determine the possibility to commercialize Voxer. The second aspect is to produce materials for supporting theory of Voxer. However, since there is limitation of equipment and software, mainly comparative testing is completed. Data and analysis acquired from the test can be used to support basic theory of Voxer from \gls{efd} point of view and to create a database to advance the study further with \gls{cfd} or with mathematical model later in the future.

Comparative testing is a process of measuring properties of the performance of the product. The primary element which constitutes an objective comparative test program is the extent to which the researchers can perform tests with independence from the manufacturers, suppliers, and marketers of the products \cite{test:book}. In this project, we performed the comparative testing on Voxer at Keravan Energy for the commissioner \gls{ltd} in order to provide data for scientific and engineering purposes; subject product to stresses and dynamic expected in use. 

\section{Measurement parameters}

The measuring device used in both experiments is Danfoss PFM 100, which specializes in measuring differential pressure on both sides of a valve in the hydronic system, which also measures $K_{v}$ factor and flow rate \cite{danfoss:web}. In our measurement, we don't use the valve needle of the measuring device but using a direct insertion of measuring hose into the medium connector to the piping system. The device needs to be calibrated every time before each measurement under static pressure influence. If the fluid circulation isn't closed, there are chances that the fluid is still moving inside pipelines which led to errors in measuring static pressure. Leakages within the system also don't let fluid completely rest, this leads to unstable static pressure calibration.
  
\section{Instruments and equipments}
The experiments focus on reconstructing the structure of the section of piping in the cooling system. The pipeline diameter is categorized as DN50. 
All of equipments and instruments are mentioned and listed in chapter \textit{Objects of Experiment} and appendix \ref{appx:third}.

\section{Data collection}

The condition of each comparative measurement presented in the following chapter was attempted to maintain stable data collection. When the data is stable after the flow has completely developed, a pressure change value, the flow rate is recorded into data sheet for each measurement. If the data doesn't achieve stable stage, a video is recorded and values are transcribed into digital data table to plot out the range of values in every measurement. Raw data is listed in appendix \ref{appx:fourth} and \ref{appx:fifth}.

\clearpage %force the next chapter to start on a new page. Keep that as the last line of your chapter!

