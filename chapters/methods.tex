% Material and Methods

\chapter{Methodology}

There are two aspects of this thesis project. First aspect is to perform a comparative testing to determine the possibility to commercialize Voxer. Second aspect is to produce materials for supporting theory of Voxer. However, since there is limitation of equipments and softwares, mainly comparative testing is able to be done. Data and analysis acquired from the test can be used to support basic theory of Voxer from Experimental Fluid Dynamics point of view \cite{springer:book}and to create database to advance the study further with Computational Fluid Dynamics or with mathematical model later in the future.

Comparative testing is a process of measuring properties of performance of the product. The primary element which constitutes an objective comparative test program is the extent to which the researchers can perform tests with independence from the manufacturers, suppliers, and marketers of the products \cite{test:book}. In this project, we performed the comparative testing on Voxer at Keravan Energy for the commissioner SansOx Ltd. in order to provide data for scientific and engineering purposes; subject product to stresses and dynamic expected in use. 

\section{Measurement parameters}

The measuring device used in both experiments is Danfoss PFM 100, which specializes in measuring differential pressure on both side of a valve in hydronic system, which also measures Kv factor and flow rate \cite{danfoss:web}. In our measurement, we don't use the valve needle of the measuring device but using direct insertion of measuring hose into medium connector to the piping system. The device needs to be calibrated every time before each measurement under static pressure influence. There is a slim chance that this method would cause minor errors or conflicts if the loop of fluid at static pressure is not closed or there are leakages along the system. 
  
\section{Instruments and equipments}
The experiments focus on reconstructing the structure of section of piping in the cooling system. The pipeline diameter is categorised as DN50. 
All of equipments and instruments are mentioned and listed in chapter \textit{Objects of Experiment} and appendices.

\clearpage %force the next chapter to start on a new page. Keep that as the last line of your chapter!

