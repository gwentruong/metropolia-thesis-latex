% Conclusions

\chapter{Discussion and Conclusion}

In this chapter, the results and analyses from the experiments are deliberated to pull a reasonable explanation for the phenomenon. Limitations and challenges from the experiments are examined as well. After that, some recommendations for the further empirical study on the subject is going to be suggested. Last but not least, a conclusion of the thesis project and possibility to commercialize Voxer are drawn.

\section{Discussion}

In the empty pipe, the introduced flow is concentrated around the center region while the flow near the inner wall is affected by \gls{noslip}. Meanwhile, Voxer disperses the flow to the wall. As observed from the result of both experiments, we can clearly recognize that having Voxer $40^{\circ}$ in any position within the pipeline led to a larger rise in pressure losses than Voxer $30^{\circ}$. This could be explained by the dimension difference between two kinds of elliptical shape. The y-axis radius of Voxer $40^{\circ}$ is much shorter than Voxer $30^{\circ}$  (see figure \vref{fig:twovox}) so the vorticity magnitude after Voxer $40^{\circ}$ is also shorter. 
\begin{wrapfigure}{r}{0.4\textwidth}
  \begin{center}
    \includegraphics[width=0.38\textwidth]{twovoxers}
  \end{center}
  \caption{Unbended Voxer 40$^{\circ}$ (up) and Voxer 30$^{\circ}$ (down)}
  \label{fig:twovox}
\end{wrapfigure}
The shorter vorticity magnitude corresponded to a higher frequency which led to higher pressure \cite{scz:article}. The appearance of Voxer $40^{\circ}$ in the mixed combination of 2 types of Voxer showed the impact of shorter vorticity magnitude on pipe pressure.

In the laboratory experiment, the contraction of sectional and total pressure losses by a single Voxer always occurs steadily in the pipelines. It can be said that each Voxer has a nominal resistance and loss reduction. With the presence of many Voxers is series, this can be argued that there was more flow resistance than loss reduction in the piping section. The first Voxer had already resulted in larger loss reduction than resistance but the following Voxers have no loss reduction because the vortex current had already been generated by the first Voxer. With the short distance between each Voxer in those combinations having more than 2 Voxers, the introduced flow wasn't completely developed before meeting the next Voxer. That Voxer became a nuisance to the water flow. Therefore, while Its nominal resistance stayed constant, the total and the larger sectional pressure difference is higher. 

However, we should also consider the bend at the outflow which caused strong turbulence at the end of the pipeline which might have increased a small amount of total pressure difference in those combinations having only the first Voxer before the starting elbow. The first Voxer reduces energy dissipation in the pipe but the ending curve at the large distance raises pressure drop. That might be a good reason for putting the second Voxer before the ending curve (combination $V14$ - Voxer $30^{\circ}$), as recommended in the technical document \cite{voxer:article}, to support loss reduction in the curved pipe. The performance of this combination was recorded as the best combination in the lab experiment. The reason for this loss reduction in pipe bends is that Voxer creates two separate spiral flow inside the tube. These two spiral flows create the kind of rolling effect in which they roll against each other like roller bearing. This effect leads to the pressure loss reduction in pipe bends.

In Keravan experiment, since the condition was uncontrolled by many minor incidents and outside factors, the evaluation must proceed in a cautious approach to avoid the overconfident conclusion. From all data collected in the experiment, especially the one illustrated in figure \vref{fig:pabdensity}, the difference of total pressure losses among empty pipe, Voxer $40^{\circ}$ and Voxer $30^{\circ}$ is very distinct. This can be explained by the disproportionation between loss reduction and resistance that was mentioned before. Looking to the sectional pressure change, which is shown in figure \vref{fig:pbdensity}, we can determine that Voxer $30^{\circ}$ has a capability to cut down energy dissipations. Since the graph has shown that Voxer $30^{\circ}$ has many values which are equal or even lower to values of the sectional pressure loss in the empty pipe. By principle, regarding of vibrations in the pipeline, Voxer wing is also expected to reduce turbulence in the sections of curved pipes, cross fittings, reducers, and valves. 

\section{Limitations and Recommendations}

There are still many limitations in our study about Voxer which are mainly related to methodology and equipment. In our experiments, we connected the pressure meter to the pipelines by measuring hose attached to a connector which doesn't have direct contact with the internal flow around the central region. It is doubting that these measuring points are only capable of measuring static pressure but not stagnation pressure. There was the centrifugal force caused by spiral flows which escalated the pressure near the inner tube surface. The original needles \cite{danfoss:web} attached to measuring hoses might result in better performance in measuring pressure drop. Since this is only a cautious thinking, it would be better to have access to other measuring instruments to compare in order to ensure the better result. 
In another hand, we could have mitigated the minor leakages problem in Keravan Energy test if we pre-pressurised the piping section before installing directly to the system. With the problem of strong pipe vibrations in a long straight pipe, it is suggested that the vibrations can be partially mitigated by switching vortex angle from $30^{\circ}$ to $40^{\circ}$  and then switching back to $30^{\circ}$ .

Regarding the nominal resistance, Voxer wing's resistance is quite insignificant, as it has neither any cross beam nor any support structures. Because of that, the only flow resistance is Voxer wing's angle which makes vortex flow. In order to lower the resistant level in the pipe, optimizing this angle and tumble finishing sharp edges of the blade should be done beforehand.

There were some time restrictions during both of our experiments which didn't enable us to perform more testings to find out more patterns of the phenomenon and relationships between different variables. They are all crucial points of Voxer wing's functions in any systems. These are some questions suggesting for further researches and testing on Voxer wing:
\begin{itemize}
  \item \textit{The relationship between Voxers distance and flow profile}. It is assured that if there are more than 2 Voxer wings in this $5\ m$ system, the flow resistance will increase. However, the reason for the variance between different combinations with only 2 Voxers hasn't been resolved yet. How far could the vortex flow travel before it becomes laminar flow? What is the distance between Voxer wings which leads to the optimum flow reduction? 
  \item \textit{The optimal design of Voxer}. Regarding some of the flow losses created by Voxer, the design of Voxer can be pondered to make it more streamlined. 
\end{itemize} 

Since the data size from comparative testing was quite restricted and depending on many outside factors, more data from more diverse combinations can be analyzed by building stimulation model of Voxer in the pipeline by using \gls{cfd}. Without limitations of time and resources, further research in the future can utilize \gls{cfd} to perform better stimulation analysis of vortex flow and vibration model by defining correct boundaries corresponding to different structures, fluid properties, model mesh, and related mathematical equations. 

There is another important remark on the credibility of the measured parameters in our experiments.  According to Bernoulli's principles, the pressure and flow velocity are inversely proportional to each other. In theory, there are a lot of factors which affects to the rise and reduction of pressure drop like the increase of flow speed, the flow resistance or the state of turbulence. In other laboratory tests conducted by \gls{ltd}, it was found that Voxer wing managed to enhance energy efficiency of the system in every measurement, regardless of the amount of Voxer and kinds of angle used in the system. Mass flow over time was measured in their tests instead of pressure drop. There are some uncertainties related to determining energy efficiency or flow losses by surveying pressure drop. Firstly, the speed boost next to the tube surface is the highest increase among other flow regions. Secondly, the impact of the rolling effect from two spiral flows created by Voxer on total pressure hasn't been investigated yet. The third uncertainty is mentioned as the impact of the centrifugal force. The last parameter needs to be considered is the rise of kinetic energy in the spiral vortex flow converted from pressure-based energy. While it contributes more mass flow, it it difficult to convert back to pressure energy when high pressure is needed. This also led the reduction of pressure condition which caused vacuum in the pipe centre which was detected in \gls{ltd}'s experiments.
\newpage

\section{Conclusion}

Comparative measurements were performed to investigate the effect of Voxer wing of \gls{ltd} on reducing energy dissipations in the water cooling piping section at Keravan Energy Ltd. The sectional and total pressure difference of the pipelines can be reduced by placing a Voxer $30^{\circ}$ before the pipe bend at the end of the piping section. This conclusion is cautiously drawn as it matches the product placement suggestion from the technical document of \gls{ltd}. In overall, Voxer can be introduced to the market with careful planning for product customization, a lot of trial testings as well as attentive supervision on product placement in customer's system. 

\clearpage %force the next chapter to start on a new page. Keep that as the last line of your chapter!
