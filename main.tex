%----------------------------------------------------------------------------------------
%	Metropolia Thesis LaTeX Template
%----------------------------------------------------------------------------------------
% License:
% This work is licensed under the Creative Commons Attribution 4.0 International License. 
% To view a copy of this license, visit http://creativecommons.org/licenses/by/4.0/.
%
% However, this license apply to this template. As a template, it is supposed to be 
% modified for your own needs (with your thesis content). For this reason, if you use 
% this project as a template and not specifically distribute it as part of a another 
% package/program, we grant the extra permission to freely copy and modify these files as 
% you see fit and even to delete this copyright notice. 
% In short, you are free to publish your thesis under whatever license you wish, even 
% keep the all rights reserved to you.
%
% Authors:
% Panu Leppäniemi, Patrik Luoto and Patrick Ausderau
%
% Credits:
% Panu Leppäniemi: abstract, def, cleaning,...
% Patrik Luoto: title page, abstract in Finnish, abbreviation, math,...
% Patrick Ausderau: initial version, style, table of content, bibliography, figure, 
%                   appendix, table, source code listing...
%
% Please:
% If you find mistakes, improve this template and alike, please contribute by sharing 
% your improvements and/or send us your feedback there: 
% https://github.com/panunu/metropolia-thesis-latex
% And of course, if you improve it, add yourself as an author.
%
% Compiler:
% Use XeLaTeX as a compiler.
 
%----------------------------------------------------------------------------------------
%	THESIS INFO
%----------------------------------------------------------------------------------------

% All general information (main language, title, author (you), degree programme, major 
% option, etc.)
% Edit the file chapters/0info.tex to change these information
% Global information (title of your thesis, your name, degree programme, major, etc.) 

\def\thesislang{finnish} %change this depending on the main language of the thesis. 
% "english" is the only other supported language currently. If someone has the swedish, please contribute!

\def\secondlang{english} %if the main language is Finnish (or Swedish), you must have 2 abstracts (one in Finnish (or Swedish) and one in English)
%If the main language is English and that you are native Finnish (or Swedish) speaker, you must have also abstract in your native language on top of the English one.

\author{Name} %your first name and last name
\def\thesis{Thesis}%keep the half based on the main thesis language
%was Opinnäytetyö

\def\alaotsikko{Subtitle} %if you don't have subtitle, empty {} it (but don't delete that line)

%Finnish section, for title/abstract
\def\otsikko{Opinnäytetyön otsikko}
\def\tutkinto{Tutkinto (esim. Insinööri (AMK))} % change to your needs, e.g. "YAMK", etc.
\def\kohjelma{Koulutusohjelma (esim. Tieto\textendash ja viestintätekniikka)}
\def\suuntautumis{Ammatillinen pääaine (esim. Mobile Solutions)}
\def\ohjaajat{
Titteli Etunimi Sukunimi\newline
Titteli Etunimi Sukunimi
}
\def\avainsanat{avainsanat}
\def\pvm{\specialdate\today}

%English section, for title/abstract
\title{Your title here}
\def\metropoliadegree {Bachelor of Engineering} % change to your needs, e.g. "master", etc.
\def\metropoliadegreeprogramme {your degree programme (e.g. Information Technology)}
\def\metropoliaspecialisation {your major option (e.g. Mobile Solutions)}
\def\metropoliainstructors {
First name Last name, Title (for example: Project Manager)\newline
First name Last name, Title (for example: Principal Lecturer)
}
\def\metropoliakeywords {Keywords}
\date{\longmonth\today}




%----------------------------------------------------------------------------------------
%	GLOBAL STYLES
%----------------------------------------------------------------------------------------

% If you need extra package, etc. modify the style/style.tex file.
% If you are using Windows OS, you will need to change default font to Arial in that 
% style/style.tex file (or install Liberation Sans font to your system).
% If you are using MacOS or linux, make sure you have Liberation Sans font installed.
% Global style. Normally should not be edited. 
% If you use windows OS, eventually change \setmainfont to Arial
% Check around commit https://github.com/panunu/metropolia-thesis-latex/commit/a0c15ac77bab1a52c59c517a18080938e57bf5ef
% to see how the font files were manually added (after downloading them: https://pagure.io/liberation-fonts/ )

\documentclass[11pt,a4paper,oneside,article]{memoir}
\usepackage[\secondlang,\thesislang]{babel}% finnish english swedish
\usepackage{iflang}
\usepackage{amsmath}
\usepackage{amsfonts}
\usepackage{amssymb}
\usepackage{fontspec}
\usepackage{tocloft}
\usepackage{titlesec}
\usepackage[hyphens]{url}
\usepackage{mathtools}
\usepackage{wallpaper}
\usepackage{datetime}
\usepackage[bookmarksdepth=subsection]{hyperref} % for automagic pdf links for toc, refs, etc.
\usepackage[amssymb]{SIunits}
\usepackage[version=3]{mhchem}
\usepackage{pgfplots} %simple plots etc
\usepackage{pgfplotstable}
\usepackage{tikz} % mindmaps, flowcharts, piecharts, examples at http://www.texample.net/tikz/examples/
\usetikzlibrary{shapes.geometric, arrows}


\renewcommand{\dateseparator}{.}
%condition for adding or not space in TOC
\usepackage{etoolbox}
%for compact list
\usepackage{enumitem}
%for block comment
\usepackage{verbatim}
%for "easier" references
\usepackage{varioref}
%forcing single line spacing in bibliography
\DisemulatePackage{setspace}
\usepackage{setspace}
%including figure (image)
\usepackage{graphicx}
\usepackage{caption}
\usepackage{subcaption}
%change the numbering for figure
\usepackage{chngcntr}
%strike trough
\usepackage{ulem}
%euro symbol
\usepackage{eurosym}
%try to count
\usepackage{totcount}
%insert source code
%\usepackage{listings}
%require -8bit -shell-escape in the xelatex compile command
\usepackage[newfloat]{minted}
\setminted{tabsize=2,linenos,breaklines,breaksymbolleft={\quad},baselinestretch=1}
\setmintedinline{breaklines}
\usepackage[justification=justified,singlelinecheck=false]{caption}
\usepackage{color}
%force the width of a table instead of column
\usepackage{tabularx}
\usepackage{booktabs} %why not booktabs? :3
% Abbreviations, acronym and glossary
\usepackage[acronym,nonumberlist,section]{glossaries}%xindy,%toc, ,nomain

\usepackage{float} % For forced figure location with modifier H (\begin{figure}[H])
\usepackage{cite} % Make citations to match Metropolia thesis guide

% change font of links in bibliography to same as other text
\usepackage{url}
\urlstyle{same}

% change punctuation of multiple cites to semicolon instead of comma: [1; 2; 3]
\renewcommand\citepunct{; }

% citep-macro for reference with period inside square brackets [1.]
\newcommand{\citep}[1]{
 \renewcommand\citeright{.]}
 \cite{#1}
 \renewcommand\citeright{]}
}

%set date format to D.M.YYYY
\newdateformat{specialdate}{\THEDAY.\THEMONTH.\THEYEAR}
%set date format to D Month YYYY
\newdateformat{longmonth}{\THEDAY~\monthname[\THEMONTH] \THEYEAR}

\newcommand\tn[1]{\textnormal{#1}} %use \tn instead of \textnormal
\newcommand\reaction[1]{\begin{equation}\ce{#1}\end{equation}} %\reaction{} for chemical reactions

%NORMAL TEXT
%all text, title, etc. in the same font: Arial
%NOTE: fontname is case-sensitive
\setmainfont{Arial}
%line space
\linespread{1.5}
\AtBeginEnvironment{tabular}{\singlespacing}
%\doublespacing
%margin
\usepackage[top=2.5cm, bottom=3cm, left=4cm, right=2cm, nofoot]{geometry}
\setlength{\parindent}{0pt} %first line of paragraph not indented
\setlength{\parskip}{16.5pt} %one empty line to separate paragraph
%list with small line space separation
\tightlists

%IMAGE - FIGURE
%the figures should be placed in the "illustration" folder
\graphicspath{{illustration/}}
%figure number without chapter (1.1, 1.2, 2.1) to (1, 2, 3)
\counterwithout{figure}{chapter}
%border around images
\setlength\fboxsep{0pt}
\setlength\fboxrule{0.5pt}
%caption font size
\captionnamefont{\small}
\captiontitlefont{\small}
%space after figure caption (and other float elements)
\setlength{\belowcaptionskip}{-7pt}

%TABLE
\counterwithout{table}{chapter}

%SOURCE CODE
\newenvironment{code}{\captionsetup{type=listing}}{}
\IfLanguageName {finnish} {\SetupFloatingEnvironment{listing}{name=Listing}} {}
%\counterwithout{lstlisting}{chapter}
%moved after begin document, otherwise does not compile

%% set this format as the default for lstlisting
%\DeclareCaptionFormat{empty}{}
%\captionsetup[lstlisting]{format=empty}

%TOC
%change toc title
\IfLanguageName {finnish} {\addto{\captionsfinnish}{\renewcommand*{\contentsname}{Contents}}} {}
%remove dots
\renewcommand*{\cftdotsep}{\cftnodots}
%chapter title and page number not in bold
\renewcommand{\cftchapterfont}{}
\renewcommand{\cftchapterpagefont}{}
%sub section in toc
\setcounter{tocdepth}{2}
%subsection numbered
\setcounter{secnumdepth}{2}
\renewcommand{\tocheadstart}{\vspace*{-15pt}}
\renewcommand{\printtoctitle}[1]{\fontsize{13pt}{13pt}\bfseries #1}
\renewcommand{\aftertoctitle}{\vspace*{-22pt}\afterchaptertitle}
%spacing afer a chapter in toc
\preto\section{%
  \ifnum\value{section}=0\addtocontents{toc}{\vskip11pt}\fi
}
%spacing afer a section in toc
\renewcommand{\cftsectionaftersnumb}{\vspace*{-3pt}}
%spacing afer a subsection in toc
\renewcommand{\cftsubsectionaftersnumb}{\vspace*{-1pt}}
%appendix in toc with "Appendix " + num
\IfLanguageName {finnish} {
  \renewcommand*{\cftappendixname}{Appendix\space}
  \renewcommand{\appendixtocname}{Appendices}
}{\renewcommand*{\cftappendixname}{Appendix\space}}
%appendix header
\IfLanguageName {finnish} {\def\appname{Appendix\space}}{\def\appname{Appendix\space}}

%TITLES
%chapter title
%\clearforchapter{\clearpage}
\titleformat{\chapter}
{\fontsize{13pt}{13pt}\bfseries\linespread{1}}%\clearpage
{\thechapter}{.5cm}{}
\titlespacing*{\chapter}{0pt}{.32cm}{9pt}
\titleformat{\section}
{\fontsize{12pt}{12pt}\linespread{1}}
{\thesection}{.5cm}{}
\titlespacing*{\section}{0pt}{14pt}{6pt}
\titleformat{\subsection}
{\fontsize{12pt}{12pt}\linespread{1}}
{\thesubsection}{.5cm}{}
\titlespacing*{\subsection}{0pt}{14pt}{6pt}


%QUOTE
\renewenvironment{quote}
  {\list{}{\rightmargin=0pt\leftmargin=1cm\topsep=-10pt}%
  \item\relax\fontsize{10pt}{10pt}\singlespacing}
  {\endlist}

%BIBLIOGRAPHY
%bibliography title to be "references"
%IF THE TITLE DON'T GET RENAMED PROPERLY, move that line after the \begin{document}
\IfLanguageName {finnish} {\addto{\captionsfinnish}{\renewcommand*{\bibname}{References}}} {\renewcommand\bibname{References}}
\makeatletter %reference list option change
\renewcommand\@biblabel[1]{#1\hspace{1cm}} %from [1] to 1 with 1cm gap
\makeatother %
\setlength{\bibitemsep}{11pt}

%count the appendices (since the chapter counter is reset after \appendix).
%! require to complie 2 times
\regtotcounter{chapter}


\makepagestyle{tiivis}
\makeevenhead{tiivis}{}{}{Tiivistelmä}
\makeoddhead{tiivis}{}{}{Tiivistelmä}

\makepagestyle{abstract}
\makeevenhead{abstract}{}{}{Abstract}
\makeoddhead{abstract}{}{}{Abstract}



% Normally, you do not need to modify the title style. It's content comes from the 
% chapters/0info.tex file.
% TITLE PAGE
% Normally, you should not edit this file.

\makeatletter
\renewcommand{\maketitle}{
\thispagestyle{empty}
\ThisCenterWallPaper{1}{viiva}
%
\vspace*{8.5cm}
\tn{\LARGE\@author\\[22pt]\Huge\IfLanguageName {finnish}{\otsikko}{\@title}\\[22pt]\LARGE\alaotsikko\\[1.75cm]}

\parbox{.7\linewidth}{
\IfLanguageName {finnish}{
  Helsinki Metropolia University of Applied Sciences\\
  \tutkinto \\
  \kohjelma \\
  \thesis\\
  \pvm
} {
  Helsinki Metropolia University of Applied Sciences\\
  \metropoliadegree \\
  \metropoliadegreeprogramme \\
  \thesis\\
  \IfLanguageName {finnish}{\pvm}{\@date} % D.M.YYYY date format for Finnish. D Month YYYY for English
}
}
\ThisLRCornerWallPaper{1}{metropolia}
%
\clearpage
}
\makeatother



%----------------------------------------------------------------------------------------
%	ABBREVIATION AND GLOSSARY
%----------------------------------------------------------------------------------------

% Add/edit all your acronyms, abbreviations, glossary entries, etc. definitions in 
% chapters/0abbr.tex file.
% You can have as many as you wish. Only the ones you use in your text (inserted with 
% \gls{} command) will print in the Glossary/Lyhenteet.
% Generate the glossary

\makeglossaries

% Acronyms, abbreviations, etc. 

\newacronym{scv}{CSV}{Comma Separated Values file}
\newacronym{pp}{PP}{Polypropylene- plastic}
\newacronym{efd}{EFD}{Experimental Fluid Dynamics}
\newacronym{cfd}{CFD}{Computational Fluid Dynamics}
\newacronym{ltd}{SansOx Ltd.}{SansOx Limited}
\newacronym{dn}{DN}{Diameter Nominal}
\newacronym{cad}{CAD}{Computer-Aided Design}


% Glossary entries

\newglossaryentry{cavit}{
	name={cavitation}, 
	description={the formation of vapour cavities in a liquid, small liquid-free zones ("bubbles" or "voids")}
}
\newglossaryentry{viscos}{
	name={viscosity}, 
	description={resistance of a fluid (liquid or gas) to a change in shape, or movement of neighbouring portions relative to one another.}
}





%----------------------------------------------------------------------------------------
%	DOCUMENT STARTS HERE...
%----------------------------------------------------------------------------------------

\begin{document}
\counterwithout{listing}{chapter}

%----------------------------------------------------------------------------------------
%	TITLE PAGE
%----------------------------------------------------------------------------------------

\input{style/title_headers.tex}
\maketitle
\newpage
%all abstract, table of content and glossary will get the metropolia logo at bottom
\LRCornerWallPaper{1}{footer}

%----------------------------------------------------------------------------------------
%	ABSTRACT / Tiivistelmä
%----------------------------------------------------------------------------------------

% If you are international student writing in English, remove the Finnish abstract.
% If you are Finnish citizen, you must have 2 abstracts, one in Finnish (or Swedish 
% depending on your mother tongue) and one in English regardless of the main language of 
% your thesis.
% \input{chapters/0abstract_fi.tex}
% Abstract in English
%Most probably, you only need to change the text of the abstract. Everything else comes from chapter/0info.tex
%If you do not have any appendix, you may delete \total{chapter} and replace with 0

\pagestyle{abstract}
\begin{otherlanguage}{english}
{\renewcommand{\arraystretch}{2}%
\begin{tabular}{ | p{4,7cm} | p{10,3cm} |}
  \hline
  Author(s) \newline
  Title \newline\newline 
  Number of Pages \newline
  Date
  & 
  \makeatletter
  \@author \newline
  \@title \newline\newline
  \pageref*{LastPage} pages + \total{chapter} appendices \newline %! if no appendices, risk to count total of chapter :D
  \IfLanguageName {finnish} {\foreignlanguage{english}{\longdate\@date}} {\@date}
  \makeatother
  \\ \hline
  Degree & \metropoliadegree
  \\ \hline
  Degree Programme & \metropoliadegreeprogramme
  \\ \hline
  Professional Major & \metropoliaspecialisation
  \\ \hline
  Instructor(s) & \metropoliainstructors
  \\ \hline
  \multicolumn{2}{|p{15cm}|}{\vspace{-22pt}
 Reducing energy dissipation inside a tube in fluid process engineering is one of the key factors to improve the efficiency of an industrial process. The purpose of this project is to demonstrate the ability to reduce fluid energy dissipation by utilizing a half elliptical blade wing named Voxer from \gls{ltd} inside a pipeline. The project was carried out by performing comparative testing on replication models of pipeline section belonging to water cooling system at Keravan Energy's biomass power plant. The Voxer inherits features from static mixer to create vortex current and has innovative characteristics to act as a cost-effective add-on solution for existing pipeline systems. 
 \newline
  
Fluid Flow experimentation on Voxer by measuring pressure drop shows that with the correct position of Voxer in the system, raising of vortex flow reduces pressure drop, which is proportional with decreasing energy losses in the fluid.
  } \\[14cm] \hline
  Keywords & \metropoliakeywords
  \\ \hline
\end{tabular}
}
\end{otherlanguage}
\clearpage



%----------------------------------------------------------------------------------------
%	License? Acknowledgement?
%----------------------------------------------------------------------------------------

% Uncomment next line and edit chapters/0license.tex if you want license in your thesis.
%\input{chapters/0license.tex}

% Uncomment next line and edit chapters/0acknowledgement.tex if you want acknowledgements.
%\input{chapters/0acknowledgement.tex}

%----------------------------------------------------------------------------------------
%	TABLE OF CONTENTS
%----------------------------------------------------------------------------------------

\input{style/toc.tex}

%list of figure, tables would come here if relevant?

%----------------------------------------------------------------------------------------
%	Lyhenteet / Abbreviation
%----------------------------------------------------------------------------------------

% If you don't use abbreviations/glossary, remove the following line.
% Abbreviation and Glossary
% Normally, you don't have to modify this file. Your abbreviations, etc. goes in 
% ../chapters/0abbr.tex file.

\begin{singlespacing}

% \gsladdall would add all terms even if not used in your text.
%\glsaddall

{
	\titleformat{\section}
	{\fontsize{13pt}{13pt}\bfseries\linespread{1}}
	{\thesection}{.5cm}{}
	%Adapt labelwidth (sorry for the ugly hack)
	\setlist[description]{leftmargin=!, labelwidth=4em}
	\IfLanguageName {finnish} {
		\printacronyms[title=Abbreviations]
	}{
		\printacronyms[title=Abbreviations]
	}
	\setlist[description]{leftmargin=!, labelwidth=7em}
	\printglossary 
	\setlist[description]{style=standard} % reset settings back to default
}
\end{singlespacing}

\clearpage


%----------------------------------------------------------------------------------------
%	CONTENT
%----------------------------------------------------------------------------------------

\input{style/content.tex}%reset page number to 1, no more logo footer, etc.

% Thesis content if you strictly follow the "Final Year Project guide". Of course, you 
% can adapt to your specific needs (add more chapter, rename them, etc.).
% Introduction

\chapter{Introduction}

Minimizing fluid energy losses is one of important missions in hydraulic engineering and chemical engineering to improve the overall efficiency of the system. The flow resistance in pipe is caused by various reasons, such as \gls{viscos}, pipe roughness or change of velocity.\newline
The new product from \gls{ltd} is an economic solution to this problem. The product is called Voxer, which consists of a Voxer wing to be inserted into a tube . The wing creates vortex flow that reduces turbulence and flow loss in the tube. Voxer can be installed in various places allocated accordingly in a whole pipeline. 

The objectives of this thesis is to present Voxer's function in reducing energy consumption in pipe and to examine the affect of different Voxer wing's combinations. The thesis was also carried out to analyze the hydraulic phenomenon and determine the Voxer's performance when introducing to the market.

The experiments were conducted in Keravan Bio-power plant of Keravan Energy Limited. Keravan Energy Ltd. is a state owned company in Kerava and Sipoo municipality. They produce electricity and district heating to not only serve their own municipality but also to sell electricity to the whole Finland. The object of this study is the water cooling piping system of the power plant.\newline
SansOx Limited is the commissioner of this project. \gls{ltd} focuses on researching, developing and marketing fresh innovation solutions for clean water market worldwide. Their goal is to advanced products in order to provide the optimal sustainable solutions for customers' needs in water treatment, process wastewater treatment, fish farming and agriculture. By executing this study, we will contribute to a completed theory profile of Voxer with quantitative research regarding to hydraulic and geometrical parameters of industrial piping system. 

In this thesis, we will demonstrate the experimental fluid dynamics setup and analysis on Voxer under laboratory condition and realistic condition in the power plant. Also, the limitation and possible further study will be discussed. 

\clearpage %force the next chapter to start on a new page. Keep that as the last line of your chapter!

% uncomment what you need.
\input{chapters/projectSpec.tex}
%% Material and Methods

\chapter{Methodology}
Fluid Flow Experiment was attempted  
\section{Measurement parameters}
\section{Instruments and equipments}

\clearpage %force the next chapter to start on a new page. Keep that as the last line of your chapter!

%\input{chapters/theory.tex}
%\input{chapters/solution.tex}
%\input{chapters/conclusion.tex}

% Sample content to demonstrate LaTeX command. You will likely delete this line and the 
% next \input{sample/*} lines. You are also safe to delete the sample/ folder and its
% content once you refershed your LaTeX skills. Also check the appendix samples.
\input{sample/1content.tex}
\input{sample/2lorem.tex}
\input{sample/3graph.tex}

%----------------------------------------------------------------------------------------
%	BIBLIOGRAPHY REFERENCES
%----------------------------------------------------------------------------------------

\input{style/biblio.tex}

%----------------------------------------------------------------------------------------
%	APPENDICES 
%----------------------------------------------------------------------------------------

\input{style/appendix.tex}
%force smaller vertical spacing in table of content
%!!! There can be some fun depending if the appendices have (sub)sections or not :D
% You will have to play with these numbers and eventually add the \vspace line  before 
% some \chapter and force another number.
% To add more fun, time to time the table of content get wrong after a build :(
\addtocontents{toc}{\vspace{11pt}}
\pretocmd{\chapter}{\addtocontents{toc}{\protect\vspace{-24pt}}}{}{}

\liite{1}% This is a hack to have right page numbering for each appendix. Make sure to 
	 % use a unique number for each appendix.
% Appendix 
% And demonstrate text references and bibliography references in appendix

\chapter{Data analysis}\label{appx:first}

\section{Mean value of data from laboratory experiment}

\begin{table}[h!]
\centering
\begin{tabular}{ | l | l | l | l | l |  }
\hline
Type & Delta P  & $P_{a}$ & $P_{b}$ & $P_{t}$ \\
\hline
None & V0 & 7.00 & 12.90 & 18.60 \\ \hline
\multirow{5}{*}{40$^{\circ}$} & V1 & 5.40 & 10.50 & 17.00 \\
 & V12 & 5.40 & 21.80 & 25.10 \\
 & V123 & 5.80 & 25.40 & 25.70 \\
 & V1234 & 5.80 & 26.60 & 27.50 \\
 & V14 & 8.60 & 9.60 & 23.00\\ \hline
\multirow{5}{*}{30$^{\circ}$} & V1 & 9.00 & 11.90 & 17.60 \\
 & V12 & 5.10 & 12.00 & 21.40 \\
 & V123 & 5.10 & 11.60 & 17.00 \\
 & V1234 & 5.10 & 11.80 & 18.10 \\
 & V14 & 6.00 & 8.70 & 14.10\\ \hline
30$^{\circ}$- 40$^{\circ}$ & V14 & 7.60 & 9.10 & 22.30 \\ \hline
\end{tabular}
\end{table}

\section{R code for analysis of Keravan experiment}
\begin{code}
  \inputminted{r}{code/test.R}
  \captionof{listing}{R code}
  \label{code:rcode}
\end{code}

\clearpage %force the next chapter/appendix to start on a new page. Keep that as the last line of your appendix!

% Sample content to demonstrate appendix in LaTeX. You
% are safe to delete this lines (and the next samples) once you refreshed your LaTeX 
% skills (and safe to delete the sample folder and all its file too).

\addtocontents{toc}{\vspace{11pt}}%fix vertical space for Table of Content
\liite{2}
\input{sample/Xappendix2.tex}

\addtocontents{toc}{\vspace{11pt}}
\liite{3}
\input{sample/X_R_example.tex}


%----------------------------------------------------------------------------------------
%	THIS IS THE END 
%----------------------------------------------------------------------------------------
\end{document}
